\documentclass[]{article}
\usepackage{lmodern}
\usepackage{amssymb,amsmath}
\usepackage{ifxetex,ifluatex}
\usepackage{fixltx2e} % provides \textsubscript
\ifnum 0\ifxetex 1\fi\ifluatex 1\fi=0 % if pdftex
  \usepackage[T1]{fontenc}
  \usepackage[utf8]{inputenc}
\else % if luatex or xelatex
  \ifxetex
    \usepackage{mathspec}
  \else
    \usepackage{fontspec}
  \fi
  \defaultfontfeatures{Ligatures=TeX,Scale=MatchLowercase}
\fi
% use upquote if available, for straight quotes in verbatim environments
\IfFileExists{upquote.sty}{\usepackage{upquote}}{}
% use microtype if available
\IfFileExists{microtype.sty}{%
\usepackage{microtype}
\UseMicrotypeSet[protrusion]{basicmath} % disable protrusion for tt fonts
}{}
\usepackage[margin=1in]{geometry}
\usepackage{hyperref}
\hypersetup{unicode=true,
            pdftitle={Eksploracja Masywnych Danych - Analiza danych},
            pdfauthor={Kajetan Zimniak \& Bartosz Górka},
            pdfborder={0 0 0},
            breaklinks=true}
\urlstyle{same}  % don't use monospace font for urls
\usepackage{color}
\usepackage{fancyvrb}
\newcommand{\VerbBar}{|}
\newcommand{\VERB}{\Verb[commandchars=\\\{\}]}
\DefineVerbatimEnvironment{Highlighting}{Verbatim}{commandchars=\\\{\}}
% Add ',fontsize=\small' for more characters per line
\usepackage{framed}
\definecolor{shadecolor}{RGB}{248,248,248}
\newenvironment{Shaded}{\begin{snugshade}}{\end{snugshade}}
\newcommand{\AlertTok}[1]{\textcolor[rgb]{0.94,0.16,0.16}{#1}}
\newcommand{\AnnotationTok}[1]{\textcolor[rgb]{0.56,0.35,0.01}{\textbf{\textit{#1}}}}
\newcommand{\AttributeTok}[1]{\textcolor[rgb]{0.77,0.63,0.00}{#1}}
\newcommand{\BaseNTok}[1]{\textcolor[rgb]{0.00,0.00,0.81}{#1}}
\newcommand{\BuiltInTok}[1]{#1}
\newcommand{\CharTok}[1]{\textcolor[rgb]{0.31,0.60,0.02}{#1}}
\newcommand{\CommentTok}[1]{\textcolor[rgb]{0.56,0.35,0.01}{\textit{#1}}}
\newcommand{\CommentVarTok}[1]{\textcolor[rgb]{0.56,0.35,0.01}{\textbf{\textit{#1}}}}
\newcommand{\ConstantTok}[1]{\textcolor[rgb]{0.00,0.00,0.00}{#1}}
\newcommand{\ControlFlowTok}[1]{\textcolor[rgb]{0.13,0.29,0.53}{\textbf{#1}}}
\newcommand{\DataTypeTok}[1]{\textcolor[rgb]{0.13,0.29,0.53}{#1}}
\newcommand{\DecValTok}[1]{\textcolor[rgb]{0.00,0.00,0.81}{#1}}
\newcommand{\DocumentationTok}[1]{\textcolor[rgb]{0.56,0.35,0.01}{\textbf{\textit{#1}}}}
\newcommand{\ErrorTok}[1]{\textcolor[rgb]{0.64,0.00,0.00}{\textbf{#1}}}
\newcommand{\ExtensionTok}[1]{#1}
\newcommand{\FloatTok}[1]{\textcolor[rgb]{0.00,0.00,0.81}{#1}}
\newcommand{\FunctionTok}[1]{\textcolor[rgb]{0.00,0.00,0.00}{#1}}
\newcommand{\ImportTok}[1]{#1}
\newcommand{\InformationTok}[1]{\textcolor[rgb]{0.56,0.35,0.01}{\textbf{\textit{#1}}}}
\newcommand{\KeywordTok}[1]{\textcolor[rgb]{0.13,0.29,0.53}{\textbf{#1}}}
\newcommand{\NormalTok}[1]{#1}
\newcommand{\OperatorTok}[1]{\textcolor[rgb]{0.81,0.36,0.00}{\textbf{#1}}}
\newcommand{\OtherTok}[1]{\textcolor[rgb]{0.56,0.35,0.01}{#1}}
\newcommand{\PreprocessorTok}[1]{\textcolor[rgb]{0.56,0.35,0.01}{\textit{#1}}}
\newcommand{\RegionMarkerTok}[1]{#1}
\newcommand{\SpecialCharTok}[1]{\textcolor[rgb]{0.00,0.00,0.00}{#1}}
\newcommand{\SpecialStringTok}[1]{\textcolor[rgb]{0.31,0.60,0.02}{#1}}
\newcommand{\StringTok}[1]{\textcolor[rgb]{0.31,0.60,0.02}{#1}}
\newcommand{\VariableTok}[1]{\textcolor[rgb]{0.00,0.00,0.00}{#1}}
\newcommand{\VerbatimStringTok}[1]{\textcolor[rgb]{0.31,0.60,0.02}{#1}}
\newcommand{\WarningTok}[1]{\textcolor[rgb]{0.56,0.35,0.01}{\textbf{\textit{#1}}}}
\usepackage{longtable,booktabs}
\usepackage{graphicx,grffile}
\makeatletter
\def\maxwidth{\ifdim\Gin@nat@width>\linewidth\linewidth\else\Gin@nat@width\fi}
\def\maxheight{\ifdim\Gin@nat@height>\textheight\textheight\else\Gin@nat@height\fi}
\makeatother
% Scale images if necessary, so that they will not overflow the page
% margins by default, and it is still possible to overwrite the defaults
% using explicit options in \includegraphics[width, height, ...]{}
\setkeys{Gin}{width=\maxwidth,height=\maxheight,keepaspectratio}
\IfFileExists{parskip.sty}{%
\usepackage{parskip}
}{% else
\setlength{\parindent}{0pt}
\setlength{\parskip}{6pt plus 2pt minus 1pt}
}
\setlength{\emergencystretch}{3em}  % prevent overfull lines
\providecommand{\tightlist}{%
  \setlength{\itemsep}{0pt}\setlength{\parskip}{0pt}}
\setcounter{secnumdepth}{0}
% Redefines (sub)paragraphs to behave more like sections
\ifx\paragraph\undefined\else
\let\oldparagraph\paragraph
\renewcommand{\paragraph}[1]{\oldparagraph{#1}\mbox{}}
\fi
\ifx\subparagraph\undefined\else
\let\oldsubparagraph\subparagraph
\renewcommand{\subparagraph}[1]{\oldsubparagraph{#1}\mbox{}}
\fi

%%% Use protect on footnotes to avoid problems with footnotes in titles
\let\rmarkdownfootnote\footnote%
\def\footnote{\protect\rmarkdownfootnote}

%%% Change title format to be more compact
\usepackage{titling}

% Create subtitle command for use in maketitle
\providecommand{\subtitle}[1]{
  \posttitle{
    \begin{center}\large#1\end{center}
    }
}

\setlength{\droptitle}{-2em}

  \title{Eksploracja Masywnych Danych - Analiza danych}
    \pretitle{\vspace{\droptitle}\centering\huge}
  \posttitle{\par}
    \author{Kajetan Zimniak \& Bartosz Górka}
    \preauthor{\centering\large\emph}
  \postauthor{\par}
      \predate{\centering\large\emph}
  \postdate{\par}
    \date{27 October, 2019}


\begin{document}
\maketitle

{
\setcounter{tocdepth}{2}
\tableofcontents
}
\hypertarget{podsumowanie-analizy}{%
\section{Podsumowanie analizy}\label{podsumowanie-analizy}}

TODO

\hypertarget{wykorzystane-biblioteki}{%
\section{Wykorzystane biblioteki}\label{wykorzystane-biblioteki}}

\begin{itemize}
\tightlist
\item
  \texttt{knitr}
\item
  \texttt{dplyr}
\item
  \texttt{tidyverse}
\end{itemize}

\hypertarget{ustawienie-ziarna-generatora}{%
\section{Ustawienie ziarna
generatora}\label{ustawienie-ziarna-generatora}}

Celem zapewnienia powtarzalności operacji losowania, a co za tym idzie
powtarzalności wyników przy każdym uruchomieniu raportu na tych samych
danych zastosowano ziarno generatora o wartości \texttt{102019}.

\begin{Shaded}
\begin{Highlighting}[]
\KeywordTok{set.seed}\NormalTok{(}\DecValTok{102019}\NormalTok{)}
\end{Highlighting}
\end{Shaded}

\hypertarget{wczytanie-danych-z-pliku}{%
\section{Wczytanie danych z pliku}\label{wczytanie-danych-z-pliku}}

Dane zamieszczone na stronie przedmiotu w postaci pliku CSV pobieramy
wyłącznie w sytuacji braku pliku w katalogu roboczym. Pozwala to nam na
ograniczenie niepotrzebnego transferu danych, jeżeli plik już istnieje.

\begin{Shaded}
\begin{Highlighting}[]
\NormalTok{file_name =}\StringTok{ "sledzie.csv"}
\NormalTok{source_url =}\StringTok{ "http://www.cs.put.poznan.pl/alabijak/emd/projekt/sledzie.csv"}

\ControlFlowTok{if}\NormalTok{ (}\OperatorTok{!}\KeywordTok{file.exists}\NormalTok{(file_name)) \{}
  \KeywordTok{download.file}\NormalTok{(source_url, }\DataTypeTok{destfile =}\NormalTok{ file_name, }\DataTypeTok{method =} \StringTok{"wget"}\NormalTok{)}
\NormalTok{\}}
\end{Highlighting}
\end{Shaded}

Po ewentualnym pobraniu wczytujemy dane do pamięci.

\begin{Shaded}
\begin{Highlighting}[]
\KeywordTok{library}\NormalTok{(}\StringTok{'knitr'}\NormalTok{)}
\KeywordTok{library}\NormalTok{(}\StringTok{'dplyr'}\NormalTok{)}
\KeywordTok{library}\NormalTok{(}\StringTok{'tidyverse'}\NormalTok{)}

\NormalTok{content =}
\StringTok{  }\NormalTok{file_name }\OperatorTok
\StringTok{  }\KeywordTok{read_csv}\NormalTok{(}\DataTypeTok{col_names =} \OtherTok{TRUE}\NormalTok{, }\DataTypeTok{na =} \KeywordTok{c}\NormalTok{(}\StringTok{""}\NormalTok{, }\StringTok{"NA"}\NormalTok{, }\StringTok{"?"}\NormalTok{)) }\OperatorTok
\StringTok{  }\KeywordTok{select}\NormalTok{(}\OperatorTok{-}\DecValTok{1}\NormalTok{)}

\NormalTok{content[}\DecValTok{0}\OperatorTok{:}\DecValTok{11}\NormalTok{] }\OperatorTok
\StringTok{  }\KeywordTok{head}\NormalTok{(}\DataTypeTok{n =} \DecValTok{6}\NormalTok{) }\OperatorTok
\StringTok{  }\KeywordTok{kable}\NormalTok{(}\DataTypeTok{align =} \StringTok{'c'}\NormalTok{, }\DataTypeTok{caption =} \StringTok{'Wybrane pomiary'}\NormalTok{)}
\end{Highlighting}
\end{Shaded}

\begin{longtable}[]{@{}ccccccccccc@{}}
\caption{Wybrane pomiary}\tabularnewline
\toprule
length & cfin1 & cfin2 & chel1 & chel2 & lcop1 & lcop2 & fbar & recr &
cumf & totaln\tabularnewline
\midrule
\endfirsthead
\toprule
length & cfin1 & cfin2 & chel1 & chel2 & lcop1 & lcop2 & fbar & recr &
cumf & totaln\tabularnewline
\midrule
\endhead
23.0 & 0.02778 & 0.27785 & 2.46875 & NA & 2.54787 & 26.35881 & 0.356 &
482831 & 0.3059879 & 267380.8\tabularnewline
22.5 & 0.02778 & 0.27785 & 2.46875 & 21.43548 & 2.54787 & 26.35881 &
0.356 & 482831 & 0.3059879 & 267380.8\tabularnewline
25.0 & 0.02778 & 0.27785 & 2.46875 & 21.43548 & 2.54787 & 26.35881 &
0.356 & 482831 & 0.3059879 & 267380.8\tabularnewline
25.5 & 0.02778 & 0.27785 & 2.46875 & 21.43548 & 2.54787 & 26.35881 &
0.356 & 482831 & 0.3059879 & 267380.8\tabularnewline
24.0 & 0.02778 & 0.27785 & 2.46875 & 21.43548 & 2.54787 & 26.35881 &
0.356 & 482831 & 0.3059879 & 267380.8\tabularnewline
22.0 & 0.02778 & 0.27785 & 2.46875 & 21.43548 & 2.54787 & NA & 0.356 &
482831 & 0.3059879 & 267380.8\tabularnewline
\bottomrule
\end{longtable}

Oryginalnie zbiór posiada znaki \texttt{?} jako oznaczenie wartości
pustej (brakującej). Dzięki wykorzystaniu parametru \texttt{na} podczas
wywołania funkcji \texttt{read\_csv} możemy zastąpić znak \texttt{?}
poprawnym oznaczeniem braku wartości \texttt{NA}.

\hypertarget{przetwarzanie-brakujux105cych-danych}{%
\section{Przetwarzanie brakujących
danych}\label{przetwarzanie-brakujux105cych-danych}}

TODO - jakieś wnioskowanie tutaj? Uśrednienie wartości?

\hypertarget{podstawowe-statystyki-zbioru-danych}{%
\section{Podstawowe statystyki zbioru
danych}\label{podstawowe-statystyki-zbioru-danych}}

W zbiorze danych mamy do czynienia z 52582 obserwacjami.


\end{document}
